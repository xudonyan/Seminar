\section{理论分析} 
%-------------------------------------------------------------------------------

\begin{frame}[t]
\frametitle{理论分析}
\framesubtitle{理论机制分析}

对于贫困地区而言,获得转移支付资金增加对地区经济增长的影响渠道,有直
接效果和间接效果两个方面
\begin{itemize}
	\item 直接效果:如何支出转移支付(激励所决定)
		\begin{itemize}
			\item 官员自身消耗
			\item 公共物品
		\end{itemize}
	\item 间接效果
		\begin{itemize}
			\item 无激励,低效率使用
			\item GDP \textuparrow ,所获转移支付 \textdownarrow ,负向
		\end{itemize}
\end{itemize}
\end{frame}

%-------------------------------------------------------------------------------






%-------------------------------------------------------------------------------

\begin{frame}[t]
	\frametitle{理论分析}
	\framesubtitle{理论模型分析}
	\begin{small}

	考虑地方政府竞争模型。$N$个地方政府,$K$单位私人资本,私人资本可以自由跨区流动,企业由私人投资,
	采用$C-D$生产函数。
	\begin{align}
	F_i=A_ik_i^\alpha P_i^\beta
	\end{align}
	地方政府:
	\begin{align}
	U_i = F_i+\lambda _i ln(c_i)\\
	P_i+c_i = S_i+tF_i
	\end{align}
	考虑无条件均等性转移支付:
	\begin{align}
	S_i^g=\sigma_i (F^0-F_i)
	\end{align}
	$F$产出,$P$公共物品,$\lambda$政府自我关心程度,$c$政府自身消费,$S$转移支付,$t$地方税率,$\tau$中央税,$\sigma$转移支付系数。
	\end{small}
\end{frame}

%-------------------------------------------------------------------------------

\begin{frame}[t]
	\frametitle{理论分析}
	\framesubtitle{理论模型分析}
	\begin{small}
		跨地区自由流动,各地区的资本净回报率都相同:
		\begin{align}
		(1-t-\tau) \frac{\partial F_i}{\partial k_i} = r
		\end{align}
		当第$i$个地方提供公共物品数量为$P_i$时,所能吸引到的私人投资$k_i$满足如下条件:
		\begin{align}
		k_i = [\frac{(1-t-\tau)\alpha A_i P_i^\beta}{r}]^{1/(1-\alpha)}
		\end{align}
		解(2)式的最大化问题,可以得到如下一阶条件:
		\begin{align}
		\frac{\partial F_i}{\partial P_i}+\frac{\partial F_i}{\partial k_i} \frac{\partial P_i}{\partial k_i} =\frac{\lambda_i}{S_i^g+tF_i-P_i+\lambda_i(t-\sigma_i)}
		\end{align}
	\end{small}
\end{frame}

%-------------------------------------------------------------------------------

\begin{frame}[t]
	\frametitle{理论分析}
	\framesubtitle{理论模型分析}
	\begin{small}
		\[
				\frac{\partial F_i}{\partial P_i}+\frac{\partial F_i}{\partial k_i} \frac{\partial P_i}{\partial k_i} =\frac{\lambda_i}{S_i^g+tF_i-P_i+\lambda_i(t-\sigma_i)}
		\]
		
		\[
			\frac{\partial F_i}{\partial P_i}+\frac{\partial F_i}{\partial k_i} \frac{\partial P_i}{\partial k_i}>0
			\]
		\[	\frac{\partial (\frac{\partial F_i}{\partial P_i}+\frac{\partial F_i}{\partial k_i} \frac{\partial P_i}{\partial k_i})}{\partial P_i}<0
		\]
		
		对(7)式进行整理可以得到:
		\begin{align}
		(t-\sigma_i)(\frac{\partial F_i}{\partial P_i}+\frac{\partial F_i}{\partial k_i} \frac{\partial P_i}{\partial k_i}) = \frac{\lambda_i}{\frac{S_i^g+tF_i-P_i}{t-\sigma_i}+\lambda_i}
		\end{align}
		
		式(8)表明,公共物品每增加1单位,其所带来的地方边际财力增加小于公共物品供给的边际成本。
	\end{small}
\end{frame}

%-------------------------------------------------------------------------------

\begin{frame}[t]
	\frametitle{理论分析}
	\framesubtitle{理论模型分析}
	\begin{small}
	由(7)式可以进一步得到:
		\begin{align}
		\frac{\partial P_i}{\partial S_i^g}= \frac{\lambda_i-\frac{\lambda_i^2}{F^0-F_i}}
		{\lambda_i [1-(t-\sigma_i)\left(\frac{\partial F_i}{\partial P_i}+\frac{\partial F_i}{\partial k_i} \frac{\partial P_i}{\partial k_i}\right)]-B^2A}
		\end{align}
		\[
		A=\frac{\partial (\frac{\partial F_i}{\partial P_i}+\frac{\partial F_i}{\partial k_i} \frac{\partial P_i}{\partial k_i})}{\partial P_i},\qquad
		B=S_i^g+tF_i-P_i+\lambda_i(t-\sigma_i)
		\]
		
		式(9)表明,当$F_i<F^0-\lambda_i$时,有$\frac{\partial P_i}{\partial S_i^g}>0$,$\frac{\partial F_i}{\partial S_i^g}>0$,进一步有:\[
		\frac{d^2P_i}{dS_i^g d\sigma_i}<0,\qquad
		\frac{d^2F_i}{dS_i^g d\sigma_i}<0
		\]
		这样,我们得到如下命题1:
		\[
		\frac{F_i}{dS_i^g}>0 \qquad
		\frac{d^2F_i}{dS_i^g d\sigma_i}<0
		\]
	\end{small}
\end{frame}

%-------------------------------------------------------------------------------

\begin{frame}[t]
	\frametitle{理论分析}
	\framesubtitle{理论模型分析}
	\begin{small}
		考虑带有配套性的有条件转移支付.配套性有条件转移支付的分配公式为:
		\begin{align}
		S_i^E = m_iP_i
		\end{align}
		可以得到如下一阶条件:
		\begin{align}
		\frac{\partial F_i}{\partial P_i}+\frac{\partial F_i}{\partial k_i} \frac{\partial P_i}{\partial k_i} =\frac{\lambda_i}{S_i^E+tF_i-P_i+\lambda_it}
		\end{align}
		由(11)式可得:
		\begin{align}
		\frac{\partial P_i}{\partial S_i^E}= \frac{(tF_i\lambda_it)\frac{\lambda_i}{P_i}}
		{\lambda_i(1-m_i) [1-m_i-t\left(\frac{\partial F_i}{\partial P_i}+\frac{\partial F_i}{\partial k_i} \frac{\partial P_i}{\partial k_i}\right)]-B^2A}
		\end{align}		

	\end{small}
\end{frame}

%-------------------------------------------------------------------------------

\begin{frame}[t]
	\frametitle{理论分析}
	\framesubtitle{理论模型分析}
	\begin{small}	
		同理可得到:
		\[
		\frac{\partial P_i}{\partial S_i^E}>0\qquad\frac{\partial F_i}{\partial S_i^E}>0\qquad\frac{d^2P_i}{dS_i^E dm_i}>0\qquad\frac{d^2F_i}{dS_i^E dm_i}>0
		\]
		
		得命题2:
		\[
		\frac{\partial F_i}{\partial S_i^E}>0\qquad\frac{d^2F_i}{dS_i^E dm_i}>0
		\]		
		
		即一个地区获得的带有配套的专项转移支付数额增加,会导致地方的经济产出增加。随着地方配套率的提高,专项转移支付数额对经济增长的边际效果会增强。
	\end{small}
\end{frame}

%-------------------------------------------------------------------------------
