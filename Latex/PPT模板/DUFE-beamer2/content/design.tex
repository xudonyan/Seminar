\section{实证策略和数据} 
%-------------------------------------------------------------------------------
\begin{frame}[t]
\frametitle{实证策略和数据}
\framesubtitle{转移支付的内生性问题}
\begin{center}
	\begin{itemize}
	\item 反向因果:上级政府可能对经济增长缺乏潜力的县给予更多的财政转移支付
	\item 遗漏变量:转移支付资金的分配也与很多不可观测因素
\end{itemize}
\end{center}
\end{frame}

%-------------------------------------------------------------------------------
\begin{frame}[t]
	\frametitle{实证策略和数据}
	\framesubtitle{国家级贫困县与转移支付资金分配}
{\small 	\begin{center}
		\begin{itemize}
			\item 1986年,首次确定了288个国家级贫困县
			\item 1993年底,中国开始实施 “八七扶贫攻坚计划”,重新确定了592个国家级贫困县名单
			\begin{itemize}
				\item[-] 1992年人均纯收入低于400元,划为新的国家级贫困县
				\item[-] 1993之前是国家
				级贫困县的,如果1992年人均纯收入不超过700元,仍保留其贫困县资格
				\item[-] 但这一标准并没有被严格实施,贫困县的认定中受到了地方政府游说的影响
			\end{itemize}
		\item 2001年对国家级贫困县再次进行了调整
		\begin{itemize}
			\item[-] 东部沿海的辽宁、山东、江苏、浙江、福建和广东六个省份的33个贫困县全部调出
			\item[-] 将西藏整体作为一个扶贫单位,单独列入计划,其原来占有的5个贫困县名额相应让出
			\item[-] 各省根据中央分配的名额数,选择省内具体的县为国家级贫困县
			\item[-] 新列入的贫困县有89个,退出名单的县也是89个
		\end{itemize}
		\end{itemize}
	\end{center}}
\end{frame}

%-------------------------------------------------------------------------------
\begin{frame}[t]
	\frametitle{实证策略和数据}
	\framesubtitle{计量模型的设定}
	国家级贫困县资格基本以1992年人均纯收入400元为界限,断点两侧的样本获得更多转移支付数额的概率发生了跳跃,因此可以使用模糊断点回归设计的策略识别因果效应。
	\begin{align}
		Transfer_{it}=c+\beta Eligible_{it}+f(z_{it})+\gama X_{it} +\lambda_p+\pi_t+v_{it}\\
		RGDP_{it} = \alpha +\tau \widehat{Transfer_{it}}+f(z_{it})+\gama X_{it} +\lambda_p+\pi _t+\mu_{it}
	\end{align}
\end{frame}

%-------------------------------------------------------------------------------
